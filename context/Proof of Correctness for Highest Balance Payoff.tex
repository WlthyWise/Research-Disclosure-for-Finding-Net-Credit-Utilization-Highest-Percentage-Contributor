\documentclass{article}
\usepackage{amsmath}
\usepackage{amsfonts}
\usepackage[left=1in, right=1in, top=1in, bottom=1in]{geometry}

% Using modern 'algorithm' and 'algpseudocode' packages
\usepackage{algorithm}
\usepackage{algpseudocode}
\usepackage{amsthm}

\title{Proof of Correctness for Minimizing Net Credit Utilization by Paying Off a Single Account}
\author{REMOVED_FOR_PRIVACY_REASONS}
\date{10/12/2025}

\begin{document}

\maketitle

\section{Problem Definition}

We are given a set of credit card accounts, which can be defined as follows:
\begin{itemize}
    \item $C = \{(b_1, l_1), \dots, (b_{n}, l_{n})\}$ is a set of tuples, where each tuple represents a single credit card account with balance $b_i$ and limit $l_i$.
    \item We assume $l_i > 0$ for all $i$, and $n > 0$.
    \item The total balance is $B_{total} = \sum_{i=1}^{n} b_i$.
    \item The total limit is $L_{total} = \sum_{i=1}^{n} l_i$.
    \item The initial net credit utilization is $U_{initial} = \frac{B_{total}}{L_{total}}$.
\end{itemize}

The objective is to find the single account $k$ which, if its balance $b_k$ were set to zero, would result in the minimum possible net credit utilization. Let $U'_{k}$ be the new utilization after paying off account $k$.
$$ U'_{k} = \frac{B_{total} - b_k}{L_{total}} $$
The goal is to prove the correctness of an algorithm that finds the account $k$ that minimizes $U'_{k}$.

\section{Algorithm}

The most efficient algorithm to achieve this goal is to identify the account with the highest balance. This can be done by sorting the accounts by balance in descending order and selecting the first element.

\begin{algorithm}
\caption{FindOptimalAccountToPayoff($C$)}
\begin{algorithmic}[1]
\Procedure{FindOptimalAccountToPayoff}{$C$}
    \Comment{Input: A set of credit card account tuples $C = \{(b_i, l_i)\}$}
    \Comment{Output: The tuple $(b_k, l_k)$ that minimizes the net utilization if paid off.}
    \State $S \gets \Call{SortByBalanceDescending}{C}$
    \State \Return $S_1$ \Comment{Return the first element, which has the highest balance.}
\EndProcedure
\end{algorithmic}
\end{algorithm}

\section{Proof of Correctness}

We will use a direct proof to show that selecting the account with the maximum balance correctly minimizes the resulting net credit utilization. A loop invariant is not necessary here because the correctness of the algorithm relies on a direct mathematical relationship rather than a step-by-step state change.

\subsection{Argument}

Let us consider any two distinct accounts from the set $C$, account $j$ and account $k$, with balances $b_j$ and $b_k$ respectively. Let's assume, without loss of generality, that account $j$ has a higher balance than account $k$.
$$ b_j > b_k $$

Let $U'_{j}$ be the resulting net utilization if account $j$ is paid off, and $U'_{k}$ be the resulting net utilization if account $k$ is paid off.
$$ U'_{j} = \frac{B_{total} - b_j}{L_{total}} $$
$$ U'_{k} = \frac{B_{total} - b_k}{L_{total}} $$

Our goal is to determine which of these two resulting utilization figures is smaller. We start with our assumption:
$$ b_j > b_k $$

Multiplying by -1 reverses the inequality:
$$ -b_j < -b_k $$

Adding the constant $B_{total}$ to both sides does not change the inequality:
$$ B_{total} - b_j < B_{total} - b_k $$

Since we assume that credit limits are positive, the total limit $L_{total} = \sum_{i=1}^{n} l_i$ must be a positive constant. Therefore, we can divide both sides by $L_{total}$ without changing the direction of the inequality:
$$ \frac{B_{total} - b_j}{L_{total}} < \frac{B_{total} - b_k}{L_{total}} $$

By substitution with our definitions of $U'_{j}$ and $U'_{k}$, we get:
$$ U'_{j} < U'_{k} $$

\subsection{Conclusion}

The result $U'_{j} < U'_{k}$ demonstrates that for any two accounts, paying off the account with the higher balance (account $j$) results in a lower final net credit utilization than paying off the account with the lower balance (account $k$).

By logical extension, this principle applies to the entire set of accounts. The account with the single **highest** balance will yield a lower resulting net utilization than any other account in the set.

Therefore, the algorithm that identifies the account with the maximum balance is correct in its approach to finding the single account that will minimize the net credit utilization percentage when paid off. The `FindOptimalAccountToPayoff` procedure, which sorts the accounts by balance in descending order and returns the first element, correctly solves the problem.

\qedsymbol

\end{document}
